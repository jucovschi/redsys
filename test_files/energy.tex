\documentclass{article}
\begin{document}
The \termref{cd=physics-energy, name=grav-potential}{gravitational
  potential energy} of a system of masses $\STRlabel[m1]{m_1}$
and $\STRlabel[m2]{M_2}$ at a distance
$\STRlabel[r]{r}$ using \termref{cd=physics-constants,
 name=grav-constant}{gravitational constant} $\STRlabel[G]{G}$
is $\STRlabel[U]{U}$
\begin{equation}
  \STRcopy{U} = -\STRcopy{G}\frac{\STRcopy{m1}\STRcopy{m2}}{\STRcopy{r}}+\STRlabel[K]{K}
\end{equation}
where \STRcopy{K} is the \termref{cd=physics-constants,
  name=integration}{constant of integration}. Choosing the convention
that \STRcopy{K}$=0$ makes calculations simpler, albeit at the cost of
making \STRcopy{U} negative.
\end{document}